%====================================================================
% LI GROUP GENERAL LATEX TEMPLATE
%=====================================================================

%------------------------------------------------------------------------------------------------------------------------
% PREAMBLE AND DOCUMENT FORMATTING
%------------------------------------------------------------------------------------------------------------------------
\documentclass[12pt]{article}

% PACKAGES
\usepackage{amsmath}			% for equation typesetting
\usepackage{amssymb}			% for equation typesetting
\usepackage{bm}                 % bold Greek letters
\usepackage{setspace}			% for 1.5 and double spacing
\usepackage{overcite}			% for superscripted in-text citations
\usepackage{url}
\usepackage{graphicx}			% main graphics package
\usepackage{wrapfig}			% allow text wrapping around figures
\usepackage[dvipsnames]{xcolor}	% for inserting colored text
%\usepackage{times}			% uncomment to use Times New Roman font
\usepackage[user=FE]{trkchg}	% % provides commands for tracking changes in compiled document
\usepackage{cancel}
\usepackage{color}
\usepackage{longtable}
\usepackage{caption}
\usepackage{titlesec}
\titleformat*{\section}{\large\bfseries}
\titleformat*{\subsection}{\normalsize\bfseries}
\titleformat*{\subsubsection}{\itshape\subsubsectionfont}
%\renewcommand{\arraystretch}{1.5}

% DOCUMENT FORMATTING
\usepackage[top=1in, bottom=1in, left=1in, right=1in]{geometry} 	% set page margins
\onehalfspacing										% uncomment for 1.5 spacing
%\doublespacing										% uncomment for double spacing
\usepackage{comment}

% BIBLIOGRAPHY FORMATTING
%\bibliographystyle{aip} 								% insert bibliography style file here
%\usepackage{subfig}%per affiancare le immagini

% CAPTION FORMATTING
\usepackage[font=footnotesize,labelfont=bf,labelsep=period,width=0.75\textwidth]{caption} 	% format single-image captions and table titles
\usepackage[font=footnotesize,labelfont=bf,labelsep=period]{subcaption} 				% format subfigure captions
\DeclareCaptionSubType*[arabic]{figure} 										% use arabic numerals for subfigure captions (e.g., 1.1, 1.2, etc.)
\DeclareCaptionLabelFormat{subfiglabel}{Figure #2} 								% append 'Figure' to subfigure captions (e.g., Figure 1.1, Figure 1.2, etc.)
%\captionsetup[subfigure]{labelformat=subfiglabel,singlelinecheck=false} 					% format subfigure captions

\graphicspath{{Figures/}}  % place figures ina subfolder

\usepackage{multirow}%To write columns that span multiple rows

\newcommand{\add}[1]{{#1}}
\newcommand{\del}[1]{{\color{red} #1}}

% CROSS-REFERENCE FORMATTING
% For use with the cleveref package
% Define the format of Figure, Table, Equation, and Section cross-references in the text
\usepackage[capitalize]{cleveref}
\crefname{figure}{Fig.}{Figs.}
\Crefname{figure}{Figure}{Figures}
\crefname{table}{Tab.}{Tabs.}
\Crefname{table}{Table}{Tables}
\crefname{equation}{Eq.}{Eqs.}
\Crefname{equation}{Equation}{Equations}
\crefname{section}{Sec.}{Secs.}
\Crefname{section}{Section}{Sections}

% USER-DEFINED COMMANDS
%% Mathematical Shortcuts
\newcommand{\pfrac}[2]{\frac{\partial #1}{\partial #2}} 				% partial derivative
\newcommand{\difrac}[2]{\frac{d #1}{d #2}} 								% derivative
\renewcommand{\Im}{\operatorname{Im}}									% imaginary symbol
\renewcommand{\Re}{\operatorname{Re}}									% real symbol
\newcommand{\dd}{\operatorname{d}}
\newcommand{\e}[1]{\operatorname{e}^{#1}}
\newcommand{\braket}[2]{\langle #1 \vert #2 \rangle} 	
\newcommand{\bra}[1]{\langle #1 \vert}
\newcommand{\ket}[1]{\vert #1 \rangle}
\newcommand{\ev}[1]{\langle #1 \rangle}
\newcommand{\evv}[1]{\langle\langle #1 \rangle\rangle}
\newcommand{\tr}{\operatorname{tr}}
\newcommand{\iu}{\ensuremath\mathrm{i}}

\begin{document}

%%%%%%%
%TITLES
%%%%%%%
\title{Let's pray Huitzilopochtli this gets published}
\author{David B. Williams-Young, Franco Egidi, Xiaosong Li$^*$ \\[12pt]
\emph{Department of Chemistry, University of Washington, Seattle, WA 98195} \\[12pt]
\texttt{email: xsli@uw.edu}}
\maketitle

\begin{abstract}
ToBeDone
\end{abstract}

\pagebreak

%%%%%%
%INTRO
%%%%%%
\section{Introduction}
TOBEDONE

%%%%%%%
%THEORY
%%%%%%%
\section{Theory and Formalism}
In this section we revisit the equations commonly encountered in the context of single Slater determinant methods, whether Hartree-Fock (HF) or Kohn-Sham Density Functional Theory (DFT), by placing the emphasis on the separation of spacial and spin coordinates.
In the following, we use the notation $\langle\cdot\rangle$ to denote the contraction of a one electron operator with the corresponding density, and $\langle\langle\cdot\rangle\rangle$ to denote the equivalent contraction for two-electron operators.

%Density Matrices
\subsection{Charge and Magnetization Densities}
The density matrix of a molecular system of $N$ electrons in a state $\Psi$ is defined as the following integral:
\begin{equation}
 \rho(x,x') = N\int\Psi(x,x_2,\dots,x_N)\Psi^*(x',x_2,\dots,x_N)\dd x_2,\dots,\dd x_N
\end{equation}
where $x=(\vec{r},\sigma)$ collects the spatial and spin coordinate of an electron.
The diagonal part of the density matrix $\rho(x)=\rho(x,x)$ is often referred to as the density function.
The presence of the spin variables within the density matrix can be made more explicit by switching to a matrix notation in terms of density components that span the space defined by the Pauli spin matrices:
\begin{gather}
 \rho = \begin{pmatrix} \rho^{\alpha\alpha} & \rho^{\alpha\beta} \\ \rho^{\beta\alpha} & \rho^{\beta\beta} \end{pmatrix} =
  \frac{1}{2}(\sigma_0n+\vec{\sigma}\cdot\vec{m}) \\
 \sigma_0 = \begin{pmatrix} 1 & 0    \\   0 & 1 \end{pmatrix} \quad;\quad
 \sigma_x = \begin{pmatrix} 0 & 1    \\   1 & 0 \end{pmatrix} \quad;\quad
 \sigma_y = \begin{pmatrix} 0 & -\iu \\ \iu & 0 \end{pmatrix} \quad;\quad
 \sigma_z = \begin{pmatrix} 0 & 1    \\   1 & 0 \end{pmatrix}
\end{gather}
We have introduced the total charge and magnetization densities $n$ and $\vec{m}$, which depend only on the spatial electron coordinates.
These quantities have dimensions of inverse volume, but their physical meaning can be restored by multiplying them by the appropriate factors:
\begin{equation}
 n\rightarrow -en \quad;\quad \vec{m}\rightarrow-\frac{1}{2}g_\mathrm{e}\mu_\mathrm{B}\vec{m}
\end{equation}
where $e$ is the elementary charge, $g_\mathrm{e}$ the electron $g$-factor, and $\mu_\mathrm{B}=e\hbar/2m_\mathrm{e}c$ is the Bohr magneton.
This separation of the density is particularly convenient, as will be shown below, for in the case of a non-collinear density, it can greatly simplify a number of equations, including the expectation values of spin operators and the expression for the Fock operator.

A similar partitioning can also be performed for those operators that possess a spin dependence.
The non-relativistic molecular Hamiltonian, in the absence of external magnetic fields, does not contain any spin operators, however if relativistic effects are introduced by means of ad-hoc operators treated perturbatively or through variationally stable two-component relativistic techniques such as the Exact Two-Component method (X2C),\cite{Liu09_219,Liu10_532,Liu09_1945,Saue09_2091} then the kinetic energy operator is replaced by its relativistic counterpart, while the electron-nuclei attraction potential is replaced by a spin-independent scalar relativistic operator, and a spin-orbit coupling term.
The one-electron part of the Hamiltonian is then partitioned as:
\begin{equation}
 h = \sigma_0 h_\mathrm{s} + \vec{\sigma}\cdot\vec{h}_\mathrm{SO}
\end{equation}
For instance in the Pauli Hamiltonian, which can be used to correct for relativistic effects perturbatively, the two terms are given by:
\begin{align}
 h_\mathrm{s} &= T + V + \frac{p^4}{8m_\mathrm{e}^3c^2} + \sum_A 2\pi\mu_\mathrm{B}^2 Z_A\delta(\vec{r}-\vec{R}_A) \\
 \vec{h}_\mathrm{SO} &= \sum_A\frac{\mu_\mathrm{B}^2}{\hbar}\frac{Z_A}{\vert\vec{r}-\vec{R}_A\vert^3}\vec{l}_A
\end{align}
Where the index $A$ runs over all nuclei and $\vec{l}_A$ is the orbital angular momentum of the electron with respect to the atom $A$.
The expectation value of the this operator is obtained by tracing it with the density.
Because the Pauli matrices are traceless, however, there are no cross terms:
\begin{equation}
 \ev{\rho h} = \frac{1}{2}\ev{nh_\mathrm{s}} + \frac{1}{2}\ev{\vec{m}\cdot\vec{h}_\mathrm{SO}}
\end{equation}
Even without relativistic effects, spin-dependent one electron interactions appear whenever other magnetic interactions are included (such as in the case of external magnetic fields).

The density matrix can be used to find expectation values of any one-electron operators, but to do the same for two-electron operators such as the simple Coulomb repulsion we need, in general, the two-electron density matrix:
\begin{equation}
 \gamma(x_1,x_1',x_2,x_2') = N(N-1)\int\Psi(x_1,x_2,\dots,x_N)\Psi^*(x_1',x_2',\dots,x_N)\dd x_3,\dots,\dd x_N
\end{equation}
In most cases there is no need to keep the primed variables and the two-electron density function $\gamma(x_1,x_2)=\gamma(x_1,x_1,x_2,x_2)$ suffices.
Also in this case the spin variables can be separated from the spatial coordinates of the electrons, but now have four spin coordinates instead of two.
A convenient way to achieve this result is to write the density as $\gamma^{\xi_1\xi_2}$ where both apices range from 0 to 3 to signify the total and magnetization parts.
This is obtained by integration of the spin variables with the Pauli matrices. Dropping the spatial variables:
\begin{equation}
\label{eq:gamma12}
 \gamma^{\xi_1\xi_2} = \int \gamma(\tau_1,\tau_1',\tau_2,\tau_2')\sigma_{\xi_1}(\tau_1,\tau_1')\sigma_{\xi_2}(\tau_2,\tau_2')\dd\tau_1\dd\tau_1'\dd\tau_2\dd\tau_2'
\end{equation}
A spin-less two-electron operator such as the Coulomb repulsion would only trace with $\gamma^{00}$:
\begin{equation}
 G = \frac{1}{2}\sum_{i\ne j}\frac{1}{r_{ij}} = \frac{1}{2}\sum_{i\ne j}g(r_{ij}) \quad\Longrightarrow\quad \ev{G} = \frac{1}{2}\evv{g\gamma^{00}}
\end{equation}
Conversely, the expectation value of the total spin operator $S^2$ for a system of $N$ electrons depends on the spin-diagonal two-electron densities instead:
\begin{equation}
\label{eq:S2}
 \ev{S^2} = \frac{3}{4}N + \frac{1}{4}\sum_{\sigma=1}^3\evv{\gamma^{\sigma\sigma}}
\end{equation}
Other examples of operators that require the spin two-electron densities include the spin-spin dipole relativistic interaction from the Breit-Pauli Hamiltonian:\cite{Dyall07_book327}
\begin{equation}
 G^{SS} = \frac{\mu_\mathrm{B}^2}{2} \sum_{i\ne j} \left(
  \frac{\vec{\sigma}_i\cdot\vec{\sigma}_j}{r_{ij}^3} -
  \frac{3(\vec{\sigma}_i\cdot\vec{r}_{ij})(\vec{\sigma}_j\cdot\vec{r}_{ij})}{r_{ij}^5} \right)
\end{equation}
The expectation value of this operator requires a combination of spin matrices, traced with the appropriate spacial operators:
\begin{equation}
 \ev{G^{SS}} = \frac{\mu_\mathrm{B}^2}{2}\sum_{\sigma=1}^3\evv{\frac{1}{r_{12}^3}\gamma^{\sigma\sigma}(\vec{r}_1,\vec{r}_2)} - 
               \frac{\mu_\mathrm{B}^2}{2}\sum_{\sigma,\tau=1}^3\evv{\frac{3r^\sigma_{12}r^\tau_{12}}{r_{12}^5}\gamma^{\sigma\tau}(\vec{r}_1,\vec{r}_2)} 
\end{equation}
Explicit expressions for these densities depend on the chosen electronic structure method, and the case of a system described by a single Slater determinant is expanded below.

%Single Determinant
\subsection{Single-Determinant Densities}
In the HF or DFT methods the density matrix can be written in terms of a single Slater determinant (for DFT the density matrix refers to the fictitious non-interacting system).
When written in terms of set of atomic-centered basis functions $\{\bm{\chi}\}$, the charge and magnetization densities can be written as:
\begin{equation}
 n(\vec{r},\vec{r}^{\,\prime}) = \bm{\chi}(\vec{r}\,)\mathbf{N}\bm{\chi}^\dagger(\vec{r}^{\,\prime}) \quad;\quad
 \vec{m}(\vec{r},\vec{r}^{\,\prime}) = \bm{\chi}(\vec{r}\,)\vec{\mathbf{M}}\bm{\chi}^\dagger(\vec{r}^{\,\prime})
\end{equation}
The matrices $\mathbf{N}$ and $\vec{\mathbf{M}}$ are obtained as the solution to the Self-Consistent Field (SCF) HF or DFT equations.
It is well known that in this case the two-body density matrix can be written in terms of the one-body density by means of the following determinant:
\begin{equation}
 \gamma(x_1,x_1',x_2,x_2') = \begin{vmatrix} \rho(x_1,x_1') & \rho(x_1,x_2') \\\rho(x_2,x_1') & \rho(x_2,x_2') \end{vmatrix}
\end{equation}
It is however interesting to look at the expressions for the spin-separated density matrices in terms of the charge and magnetization densities.
%For instance, in the spin-diagonal case, we have:
%\begin{align}
% \gamma^{00} &=
%  \frac{1}{2}\big(2n(r_1,r_1')n(r_2,r_2')-n(r_1,r_2')n(r_2,r_1')\big) - \frac{1}{2}\vec{m}(r_1,r_2')\cdot\vec{m}(r_2,r_1') \\
% \gamma^{\sigma\sigma} &=
%  m_i(r_1,r_1')m_i(r_2,r_2') - \frac{1}{2}n(r_1,r_2')n(r_2,r_1') + \frac{1}{2}\vec{m}(r_1,r_2')\cdot\vec{m}(r_2,r_1') - m_i(r_1,r_2')m_i(r_2,r_1') \\
% \sum_{\sigma=x,y,z}\gamma^{\sigma\sigma} &=
%  \vec{m}(r_1,r_1')\cdot\vec{m}(r_2,r_2') - \frac{3}{2}n(r_1,r_2')n(r_2,r_1') + \frac{1}{2}\vec{m}(r_1,r_2')\cdot\vec{m}(r_2,r_1')
%\end{align}
If we define $m_0=\iu n$ then it can be shown using \cref{eq:gamma12} that the spin densities can be written as:
\begin{align}
 \gamma^{\sigma\sigma} &= \begin{vmatrix} m_\sigma(r_1,r_1') & m_\sigma(r_1,r_2') \\ m_\sigma(r_2,r_1') & m_\sigma(r_2,r_2') \end{vmatrix} 
  + \frac{1}{2} \sum_\tau m_\tau(r_1,r_2')m_\tau(r_2,r_1') \\
 \gamma^{\sigma\tau}+\gamma^{\tau\sigma} &= \begin{vmatrix} m_\sigma(r_1,r_1') & m_\tau(r_1,r_2')   \\ m_\sigma(r_2,r_1') & m_\tau(r_2,r_2')   \end{vmatrix} +
                                            \begin{vmatrix} m_\tau(r_1,r_1')   & m_\sigma(r_1,r_2') \\ m_\tau(r_2,r_1')   & m_\sigma(r_2,r_2') \end{vmatrix} \\
 \gamma^{\sigma\tau}-\gamma^{\tau\sigma} &= \begin{vmatrix} m_\sigma(r_1,r_1') & m_\zeta(r_1,r_2')  \\ m_\eta(r_2,r_1')   & m_\tau(r_2,r_2')   \end{vmatrix} -
                                            \begin{vmatrix} m_\tau(r_1,r_1')   & m_\eta(r_1,r_2')   \\ m_\zeta(r_2,r_1')  & m_\sigma(r_2,r_2') \end{vmatrix} \quad(\varepsilon_{\sigma\tau\zeta\eta}=+1)
\end{align}
Where $\varepsilon_{\sigma\tau\zeta\eta}$ is the rank-four Levi-Civita symbol, and we have isolated the isotropic, symmetric anisotropic, and antisymmetric anisotropic spin components of the two-electron density.
These are the components that are actually needed in all systems without a preferential spin orientation with respect to laboratory frame of reference.
The isotropic and symmetric anisotropic components are used to calculate the spin-spin dipole couplings, the antisymmetric anisotropy can instead appear in other types of terms, such as the Dzyaloshinskii-Moriya interaction,\cite{Dzyaloshinsky58_241,Moriya60_91} which involves the cross products of two spin dipole moments.
This way of isolating density components can provide useful insights into the computational methods used to solve two-component problems as will be shown below.

Using these expressions it is possible to easily derive explicit expressions for expectation values of spin operators.
For instance, \cref{eq:S2} becomes:
\begin{equation}
\begin{split}
 \ev{S^2} &= \frac{3}{4}N + \frac{1}{4}(\tr\vec{\mathbf{M}}\mathbf{S})^2
          - \frac{3}{8}\tr\mathbf{NSNS} + \frac{1}{8}\tr\vec{\mathbf{M}}\mathbf{S}\cdot\vec{\mathbf{M}}\mathbf{S} =\\
          &= \frac{1}{4}(\tr\vec{\mathbf{M}}\mathbf{S})^2 + \frac{1}{2}\tr\vec{\mathbf{M}}\mathbf{S}\cdot\vec{\mathbf{M}}\mathbf{S}
\end{split}
\end{equation}
Where $\mathbf{S}$ is the AO overlap matrix, and to simplify the expression we have invoked the idempotency condition:
\begin{equation}
 \mathbf{PSP} = \mathbf{P} \quad\Longrightarrow\quad
 2\mathbf{N} = \mathbf{NSN} + \vec{\mathbf{M}}\mathbf{S}\cdot\vec{\mathbf{M}} \quad;\quad
 2\vec{\mathbf{M}} = \mathbf{NS}\vec{\mathbf{M}} + \vec{\mathbf{M}}\mathbf{SN} + \iu \vec{\mathbf{M}}\mathbf{S}\times\vec{\mathbf{M}}
\end{equation}

%Energy
\subsection{Energy and Fock Operator}
If we consider a Hamiltonian containing scalar and spin-orbit one-electron terms, and the bare Coulomb two-electron repulsion operator.
Given a one-electron density matrix $\rho$, the mean-field energy would be given as:
\begin{equation}
 E = \frac{1}{2}\ev{nh_\mathrm{s}} + \frac{1}{2}\ev{\vec{m}\cdot\vec{h}_\mathrm{SO}} + \frac{1}{2}\evv{\gamma^{00}g}
\end{equation}
The Fock operator can be found by differentiating this energy with respect to the density matrix components.
The derivatives with respect to the charge and magnetization densities can be taken separately:
\begin{equation}
 \frac{\delta}{\delta\rho} = 
  \begin{pmatrix} \frac{\delta}{\delta\rho^{\alpha\alpha}} & \frac{\delta}{\delta\rho^{\alpha\beta}} \\
                  \frac{\delta}{\delta\rho^{\beta\alpha}}  & \frac{\delta}{\delta\rho^{\beta\beta}} \end{pmatrix} =
  \frac{\delta n}{\delta\rho}\frac{\delta}{\delta n} + \frac{\delta \vec{m}}{\delta\rho}\cdot\frac{\delta}{\delta \vec{m}} = 
    \sigma_0\frac{\delta}{\delta n} + \vec{\sigma}\cdot\frac{\delta}{\delta\vec{m}}
\end{equation}
The well-known expression for the Fock operator which, in the AO basis, is then given by:
\begin{equation}
 \mathbf{F}(\mathbf{P}) = \pfrac{E}{\mathbf{P}^\mathrm{t}} = \frac{1}{2}\sigma_0\big(\mathbf{h}_\mathrm{s} + 2\mathbf{J}(\mathbf{N}) - \mathbf{K}(\mathbf{N})\big) +
  \frac{1}{2}\vec{\sigma}\cdot\big(\vec{\mathbf{h}}_\mathrm{SO} - \mathbf{K}(\vec{\mathbf{M}})\big)
\end{equation}
Where the Coulomb and Exchange contributions are directly derived from the expression of two-electron density matrix:
\begin{equation}
 \gamma^{00} = \frac{1}{2}\big(2n(r_1,r_1')n(r_2,r_2')-n(r_1,r_2')n(r_2,r_1')\big) - \frac{1}{2}\vec{m}(r_1,r_2')\cdot\vec{m}(r_2,r_1') 
\end{equation}
In the case of DFT the exchange is substituted with the exchange-correlation potential and magnetic field.
This expression for the two-component Fock operator can be immediately be reduced to the Restricted and Unrestricted limiting cases (in the absence of spin-orbit interactions).
The UHF form is obtained by setting $\mathbf{M}_x$ and $\mathbf{M}_y$ to zero, while for RHF all magnetization components are zero, and the familiar $2\mathbf{J} - \mathbf{K}$ form of two electron part is already manifest.


%%%%%%%%%%%%
%SCF METHODS
%%%%%%%%%%%%
\section{Self-Consistent Field Methods}

%%%%%%%%%%%%
%CONCLUSIONS
%%%%%%%%%%%%
\section{Conclusions and Perspectives}
ToBeDone

\section{Acknowledgements}
ToBeDone

\newpage
\bibliographystyle{jcp}
\bibliography{Journal_Short_Name.bib,Li_Group_References.bib,Egidi_References.bib,All_References.bib}

\end{document}
